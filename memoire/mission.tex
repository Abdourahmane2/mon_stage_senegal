\subsection{Contexte du stage}
Les universités Africaines plus précisément l'université cheikh Anta Diop de Dakar font face a une multitude de problèmes une croissance rapide des effectifs étudiants, des inscription de plus en plus tardives, des taux d'abandon élevés et une gestion des ressources académiques souvent inefficace liées a une manque de ressources financières et humaines. 
Dans ce contexte, l'analyse des données devient un outil essentiel pour comprendre les tendances d'inscription, les performances académiques et anticiper les besoins futurs.
C'est dans ce cadre que j'ai effectué mon stage au sein de la Division des Études Statistiques (DES) pour essayer a travers l'analyse des données d'apporter des solutions aux problèmes rencontrés par l'université.
\subsection{Objectifs du stage}
L'objectif principal de mon stage était de contribuer à l'analyse des données relatives aux étudiants de l'université. Plus précisément, il s'agissait de :
\begin{itemize}
    \item collecter,nettoyer et structurer les données historiques sur les effectifs des étudiants.
    \item Réaliser des analyses descriptives pour identifier les tendances les variations saisonnières et les anomalies dans les données.
    \item Développer un modèle de machine learning facilitant
    ainsi la planification stratégique et la prise de décision au sein de l'université. 
    \item Présenter les analyses et recommandations de manière claire décideurs de l'université 
\end{itemize}
\subsection{Difficultés rencontrées}
Au cours de mon stage, j'ai rencontré plusieurs difficultés :
\begin{itemize}
    \item \textbf{Qualité des données} : Absence de texte pour expliquer explicitement les colonnes dans la base de données, ce qui a rendu la compréhension des données plus difficile.
    \item \textbf{Accès aux données} : Difficultés à accéder à certaines données historiques en raison de la structure des bases de données et des restrictions d'accès.
    \item \textbf{Complexité des analyses} : Avec plusieurs faculté , départements et ecoles les données étaient volumineuses et complexes, nécessitant des techniques avancées de nettoyage et d'analyse.
    \item \textbf{Intégration des données} : Fusionner les données provenant de différentes sources a été un défi en raison de la diversité des formats et des structures.
    \end{itemize}