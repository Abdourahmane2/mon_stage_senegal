
L'université cheikh Anta Diop produit chaque année une quantité importante de données relatives aux étudiants.  
ces données sont essentielles pour la gestion et la planification des ressources académiques.l'analyse de ses donnes peuvent permettre a l'université 
de tirer des enseignements précieux sur les tendances d'inscription, les performances académiques et anticiper les besoins futures . 

C'est dans ce contexte que j'effectue mon stage au sein de la Division des Études Statistiques (DES) de cette université, 
une structure charge de réalise des analyses statistiques et de fournir des informations pertinentes pour la prise de décision au sein de l'université.

Durant ce stage  j'ai ete implique dans diverses taches allant de la compréhension des donnes a l'implementation d'un modèle de machine learning . 
Cette expérience m'a permis de développer mes compétences en analyse de données, en programmation et en visualisation des données tout en découvrant la problématique des universités africaines .

Ce Rapport présente en details les objectifs de mon stage, les technologies utilisées, les résultats obtenus  ainsi que les competences acquises.il proposera quelques pistes de réflexion pour l'amélioration continue de l'analyse des données au sein de l'université.


